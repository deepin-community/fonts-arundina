%%
%% Test standard sentences for every fonts
%% 2004/02/06 Poonlap Veerathanabutr <poonlap@linux.thai.net>
%% 	      - rewrite this file
%% 2011/12/26 Theppitak Karoonboonyanan <thep@linux.thai.net>
%%            - rewrite using macros
%%
\documentclass[a4paper]{article}
\usepackage[english,thai]{babel}
\usepackage[utf8x]{inputenc}
\usepackage[scale=0.65]{fonts-arundina}

\newcommand{\testthaipoem}[3]{%
  \usefont{LTH}{#1}{#2}{#3}
  \noindent
  \begin{tabbing}
  {\usefont{LTH}{arunserif}{b}{n}
  XXXXXXXXXXXXXXXXXXXXXXXXX} \=
  {\usefont{LTH}{arunserif}{b}{n}
  XXXXXXXXXXXXXXXXXXXXXXXXX}\kill
  \hspace{1em}๏ เป็นมนุษย์สุดประเสริฐเลิศคุณค่า \> กว่าบรรดาฝูงสัตว์เดรัจฉาน \\
  จงฝ่าฟันพัฒนาวิชาการ          \> อย่าล้างผลาญฤๅเข่นฆ่าบีฑาใคร\\
  ไม่ถือโทษโกรธแช่งซัดฮึดฮัดด่า    \> หัดอภัยเหมือนกีฬาอัชฌาสัย \\
  ปฏิบัติประพฤติกฎกำหนดใจ       \> พูดจาให้จ๊ะๆ จ๋าๆ น่าฟังเอย ฯ\\
  \end{tabbing}}

\newcommand{\testenglish}[3]{%
  \usefont{LTH}{#1}{#2}{#3}
  \noindent
  A quick brown fox jumps over the lazy dog.}

\newcommand{\testEnglish}[3]{%
  \usefont{LTH}{#1}{#2}{#3}
  \noindent
  \MakeUppercase{A quick brown fox jumps over the lazy dog.}}

\newcommand{\testligkern}[3]{%
  \usefont{LTH}{#1}{#2}{#3}
  \noindent
  ที่ ท่า ทิ้ง ท้า กิ๊ง ก๊ง ตี๋ ต๋า บ่น ป่น, บ้น ป้น, บ๊น ป๊น, บ๋น ป๋น บิน ปิน บีน ปีน บิ่น ปิ่น บัน ปั่น บั่น
  ก็ ป็ ปู่ ญ ญุ ญู ญฺ ฐ ฐุ ฐู ฐฺ กุ ฎุ ฎู ฎฺ ฏุ ฏู ฏฺ บำ บ่ำ ปำ ป่ำ -\textyamakkan{}
  \textfongmun{} \textangkhankhu{} \textkhomut{}
  - -- --- `` '' \dag{} \ddag{} \S{} \P{} \${} \ae{} \AE{} \oe{} \OE{} \aa{}
  \AA{} \ss{} \copyright{} \textregistered{} \texttrademark{} \textbackslash{}
  \textasciicircum{} \textasciitilde{} \textbar{} \textbraceleft{}
  \textbraceright{} ?` !` ff fi fl ffi ffl tt ti AV\\}

\newcommand{\testpali}[3]{%
  \usefont{LTH}{#1}{#2}{#3}
  \noindent
  \textpali{หตฺเถสุ ภิกฺขเว สติ, อาทานนิกฺเขปนํ ปญฺญายติ}\\
  \textpali{เอวเมว โข ภิกฺขเว}\\
  \textpali{จกฺขุสมิํปิ สติ}\\
  \textpali{จกฺขุสมฺผสฺสปจฺจยา อุปฺปชฺชติ อชฺฌตฺตํ สุขทุกฺขํ}\\
  \textpali{ทิฏฺฐา มยา ภิกฺขเว ฉ ผสฺสายตนิกา นาม นิรยา}\\}


\begin{document}
\pagestyle{empty}
\vfil
\begin{figure*}
\Huge
\hspace*{.2\textwidth}\usefont{LTH}{arunserif}{m}{n}แบบอักษรไทยใน \LaTeX\\
\hspace*{.2\textwidth}\usefont{LTH}{arunsans}{m}{n}แบบอักษรไทยใน \LaTeX\\
\hspace*{.2\textwidth}\usefont{LTH}{arunmono}{m}{n}แบบอักษรไทยใน \LaTeX\\
\end{figure*}
\vfil
\clearpage

\pagestyle{plain}
\section{\usefont{LTH}{arunserif}{b}{n}Arundina Serif}
\subsection{ตัวอย่างประโยคภาษาไทย\protect\footnote{โดยสมาคมคอมพิวเตอร์แห่งประเทศไทยในพระบรมราชูปถัมภ์}}

\testthaipoem{arunserif}{m}{n}

\testthaipoem{arunserif}{b}{n}

\testthaipoem{arunserif}{m}{it}

\testthaipoem{arunserif}{b}{it}

\subsection{ตัวอย่างภาษาอังกฤษ}

\testenglish{arunserif}{m}{n}

\testenglish{arunserif}{b}{n}

\testenglish{arunserif}{m}{it}

\testenglish{arunserif}{b}{it}

\testEnglish{arunserif}{m}{n}

\testEnglish{arunserif}{b}{n}

\testEnglish{arunserif}{m}{it}

\testEnglish{arunserif}{b}{it}


\subsection{การจัดระดับตัวอักษรและตัวอักษรพิเศษ}

\testligkern{arunserif}{m}{n}

\testligkern{arunserif}{b}{n}

\testligkern{arunserif}{m}{it}

\testligkern{arunserif}{b}{it}


\subsection{ภาษาบาลี-สันสกฤต}

\testpali{arunserif}{m}{n}

\testpali{arunserif}{b}{n}

\testpali{arunserif}{m}{it}

\testpali{arunserif}{b}{it}

\vfil\pagebreak


\section{\usefont{LTH}{arunsans}{b}{n}Arundina Sans}

\subsection{ตัวอย่างประโยคภาษาไทย}

\testthaipoem{arunsans}{m}{n}

\testthaipoem{arunsans}{b}{n}

\testthaipoem{arunsans}{m}{it}

\testthaipoem{arunsans}{b}{it}


\subsection{ตัวอย่างภาษาอังกฤษ}

\testenglish{arunsans}{m}{n}

\testenglish{arunsans}{b}{n}

\testenglish{arunsans}{m}{it}

\testenglish{arunsans}{b}{it}

\testEnglish{arunsans}{m}{n}

\testEnglish{arunsans}{b}{n}

\testEnglish{arunsans}{m}{it}

\testEnglish{arunsans}{b}{it}


\subsection{การจัดระดับตัวอักษรและตัวอักษรพิเศษ}

\testligkern{arunsans}{m}{n}

\testligkern{arunsans}{b}{n}

\testligkern{arunsans}{m}{it}

\testligkern{arunsans}{b}{it}


\subsection{ภาษาบาลี-สันสกฤต}

\testpali{arunsans}{m}{n}

\testpali{arunsans}{b}{n}

\testpali{arunsans}{m}{it}

\testpali{arunsans}{b}{it}

\vfil\pagebreak

\section{\usefont{LTH}{arunmono}{b}{n}Arundina Sans Mono}

\subsection{ตัวอย่างประโยคภาษาไทย}

\testthaipoem{arunmono}{m}{n}

\testthaipoem{arunmono}{b}{n}

\testthaipoem{arunmono}{m}{it}

\testthaipoem{arunmono}{b}{it}

\subsection{ตัวอย่างภาษาอังกฤษ}

\testenglish{arunmono}{m}{n}

\testenglish{arunmono}{b}{n}

\testenglish{arunmono}{m}{it}

\testenglish{arunmono}{b}{it}

\testEnglish{arunmono}{m}{n}

\testEnglish{arunmono}{b}{n}

\testEnglish{arunmono}{m}{it}

\testEnglish{arunmono}{b}{it}


\subsection{การจัดระดับตัวอักษรและตัวอักษรพิเศษ}

\testligkern{arunmono}{m}{n}

\testligkern{arunmono}{b}{n}

\testligkern{arunmono}{m}{it}

\testligkern{arunmono}{b}{it}


\subsection{ภาษาบาลี-สันสกฤต}

\testpali{arunmono}{m}{n}

\testpali{arunmono}{b}{n}

\testpali{arunmono}{m}{it}

\testpali{arunmono}{b}{it}

\vfil\pagebreak


\end{document}
